\documentclass[10pt, a4paper]{article}

\usepackage{mathtext}
\usepackage[T1,T2A]{fontenc}
\DeclareSymbolFont{T2Aletters}{T2A}{cmr}{m}{it} 
\usepackage[utf8]{inputenc}
\usepackage[english,russian]{babel}
\usepackage{graphicx,amsmath,color,listings}
\usepackage[margin=1in]{geometry}

\title{Домашнее задание №7}
\author{Андрей Козлов, гр. 4538}
\date{}

\lstnewenvironment{sql}
{\lstset{
	language=SQL,
	frame=single,
	escapeinside={\%*}{*)}
}}
{}

\begin{document}
\maketitle

Пусть используются таблицы со следующими именами:
\begin{itemize}
	\item students
	\item groups
	\item lecturers
	\item courses
	\item marks
	\item grouplists
	\item schedule
\end{itemize}

\begin{enumerate}

	\item {Напишите запрос, удаляющий всех студентов, не имеющих долгов.

	\begin{sql}
DELETE FROM grouplists WHERE student_id NOT IN
	(SELECT student_id FROM marks WHERE points < 60);
	\end{sql}
	}

	\item {Напишите запрос, удаляющий всех студентов, имеющих 3 и более долгов.

	\begin{sql}
DELETE FROM grouplists WHERE student_id IN
	(SELECT student_id FROM marks WHERE points < 60
	GROUP BY student_id HAVING count(*) >= 3);
	\end{sql}
	}

	\item {Напишите запрос, удаляющий все группы, в которых нет студентов.

	\begin{sql}
DELETE FROM groups WHERE group_id NOT IN
	(SELECT group_id FROM grouplists);
	\end{sql}
	}

	\item {Создайте view \textsf{Losers} в котором для каждого студента, имеющего долги указано их количество.

	\begin{sql}
CREATE VIEW Losers AS SELECT student_name, count(*) FROM
	marks NATURAL JOIN students NATURAL JOIN grouplists
		WHERE points < 60 GROUP BY student_name;
	\end{sql}
	}

	\item {Создайте таблицу \textsf{LoserT}, в которой содержится та же информация, что во view \textsf{Losers}. Эта таблица должна автоматически обновляться при изменении таблицы с баллами.

	\begin{sql}
CREATE MATERIALIZED VIEW LosersT AS
	(SELECT student_name, count(*) FROM
	marks NATURAL JOIN students NATURAL JOIN grouplists
		WHERE points < 60 GROUP BY student_name);

CREATE FUNCTION refresh_loserst() RETURNS trigger AS $refresh_loserst$ 
BEGIN
	REFRESH MATERIALIZED VIEW LosersT;
	RETURN NEW;
END; 
$refresh_loserst$ LANGUAGE plpgsql;

CREATE TRIGGER refresh_loserst AFTER UPDATE ON marks
	EXECUTE PROCEDURE refresh_loserst();
	\end{sql}
	}

	\item {Отключите автоматическое обновление таблицы \textsf{LoserT}.

	\begin{sql}
DROP FUNCTION IF EXISTS refresh_loserst() CASCADE;
	\end{sql}
	}

	\item {Напишите запрос (один), которой обновляет таблицу \textsf{LoserT}, используя данные из таблицы \textsf{NewPoints}, в которой содержится информация о баллах, проставленных за последний день.

\begin{sql}
UPDATE marks SET points = NewPoints.points
	WHERE marks.student_id = NewPoints.student_id
	AND marks.course_id = NewPoints.course_id;
\end{sql}
	}

	\item {Добавьте проверку того, что все студенты одной группы изучают один и тот же набор курсов.

\begin{sql}

\end{sql}
	}

	\item {Создайте триггер, не позволяющий уменьшить баллы студента по предмету. При попытке такого изменения, баллы изменяться не должны.

	\begin{sql}
CREATE FUNCTION not_decrease_mark() RETURNS trigger AS $decrease_mark$ 
BEGIN 
	IF (NEW.points < OLD.points) THEN	
		RAISE EXCEPTION 'student % already has % points',
		NEW.student_id, OLD.points;
	END IF;
	RETURN NEW;
END; 
$decrease_mark$ LANGUAGE plpgsql;

CREATE TRIGGER not_decrease_marks BEFORE UPDATE ON marks
	FOR EACH ROW EXECUTE PROCEDURE not_decrease_mark();
	\end{sql}
	}

\end{enumerate}

\end{document}