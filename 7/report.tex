\documentclass[10pt, a4paper]{article}

\usepackage{mathtext}
\usepackage[T1,T2A]{fontenc}
\DeclareSymbolFont{T2Aletters}{T2A}{cmr}{m}{it} 
\usepackage[utf8]{inputenc}
\usepackage[english,russian]{babel}
\usepackage{graphicx,amsmath,color,listings}
\usepackage[margin=1in]{geometry}

\title{}
\author{Андрей Козлов, гр. 4538}
\date{}

\lstnewenvironment{sql}
{\lstset{
	language=SQL,
	frame=single,
	escapeinside={\%*}{*)}
}}
{}

\begin{document}
\maketitle

Пусть:
\begin{itemize}
	\item S :: students
	\item G :: groups
	\item L :: lecturers
	\item C :: courses
	\item M :: marks
	\item GL :: grouplists
	\item SCH :: schedule
\end{itemize}

\begin{enumerate}

	\item {Напишите запрос, удаляющий всех студентов, не имеющих долгов.

	\begin{sql}
DELETE FROM grouplists WHERE student_id NOT IN
	(SELECT student_id FROM marks WHERE points < 60);
	\end{sql}
	}

	\item {Напишите запрос, удаляющий всех студентов, имеющих 3 и более долгов.

	\begin{sql}
DELETE FROM grouplists WHERE student_id IN
	(SELECT student_id FROM marks WHERE points < 60
	GROUP BY student_id HAVING count(*) >= 3);
	\end{sql}
	}

	\item {Напишите запрос, удаляющий все группы, в которых нет студентов.

	\begin{sql}
DELETE FROM groups WHERE group_id NOT IN
	(SELECT group_id FROM grouplists);
	\end{sql}
	}

	\item {Создайте view Losers в котором для каждого студента, имеющего долги указано их количество.

	\begin{sql}
CREATE VIEW Losers AS SELECT student_name, count(*) FROM
	marks NATURAL JOIN students NATURAL JOIN grouplists
		WHERE points < 60 GROUP BY student_name;
	\end{sql}
	}

	\item {Создайте таблицу LoserT, в которой содержится та же информация, что во view Losers. Эта таблица должна автоматически обновляться при изменении таблицы с баллами.

	}

	\item {Отключите автоматическое обновление таблицы LoserT.

	}

	\item {Напишите запрос (один), которой обновляет таблицу LoserT, используя данные из таблицы NewPoints, в которой содержится информация о баллах, проставленных за последний день.

	}

	\item {Добавьте проверку того, что все студенты одной группы изучают один и тот же набор курсов.

	}

	\item {Создайте триггер, не позволяющий уменьшить баллы студента по предмету. При попытке такого изменения, баллы изменяться не должны.

	\begin{sql}
CREATE FUNCTION not_decrease_mark() RETURNS trigger AS $decrease_mark$ 
BEGIN 
	IF (NEW.points < OLD.points) THEN	
		RAISE EXCEPTION 'student % already has % points',
		NEW.student_id, OLD.points;
	END IF;
	RETURN NEW;
END; 
$decrease_mark$ LANGUAGE plpgsql;

CREATE TRIGGER not_decrease_marks BEFORE UPDATE ON marks
	FOR EACH ROW EXECUTE PROCEDURE not_decrease_mark();
	\end{sql}
	}

\end{enumerate}

\end{document}